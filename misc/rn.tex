\input{words}
\def\build{{\sf BUILD}}
\documentstyle{article}
\begin{document}
\title{Installing and Using \build}
\author{\Rich\\
        \Digital\\
	\AITG\\
        \Street\\
	\City\\
        \Phone\\
        \Arpa}
\maketitle

\section{Tape Contents}

The \build\ distribution tape was created using the Symbolics {\em carry-tape}
system.  The {\tt tape:carry-list} and {\tt tape:carry-load} functions should
be used to manipulate the tape.  The files on the tape are arranged in three
directories:
\begin{description}
\item[{\tt :>robbins}]
This directory contains a file that shows how
a user might wish to load \build\ into a Lisp environment and a file
that includes several \build\ {\em system models}.

\item[{\tt :>robbins>build}] This directory contains the source files for
\build.

\item[{\tt :>robbins>build>test}]
This directory contains all of the files referenced in the sample
system models.
\end{description}

\section{Installing \build}

The {\tt tape:carry-load} function should be used to install the contents of
the \build\ distribution tape.  It is recommended that the files be installed
in three directories, divided in the fashion that they have been placed on 
the tape.  
Once \build\ has been installed, several files will need to be modified
before a working version of \build\ can be loaded into a Lisp environment.
\begin{description}

\item[\tt test-models.lisp]  All of the directory references in this file
should be changed to point to the directory where the test models are installed.
There are seven system models; each with one reference to modify.

\item[\tt lispm-init.lisp]  This file contains (among other things) a 
form that loads the file {\tt build.system}.  The call to {\tt load} should
be modified to refer to the directory where {\tt build.system} is located.

\item[\tt build.system]  \build\ is compiled and loaded by the Symbolics
{\em defsystem} utility.  The file {\tt build.system}
contains the forms used by {\tt defsystem} and the forms to set up the package
that \build\ resides in.  Modify the value of the {\tt :pathname-default}
keyword in the two calls to {\tt defsystem} to refer to the
directory where the source files for \build\ are located.

\end{description}

\section{Loading \build\ Into The Lisp Environment}

This file {\tt lispm-init.lisp} contains function calls that will
load \build\ and import the user visible entities from the {\tt BUILD} package
into the {\tt USER} package.  When the init file is loaded the user will be
asked if \build\ should be loaded; there are four choices: Yes, No, Recompile,
and Quiet.
Yes and No do the obvious things.  Quiet is just like Yes except that the system
will not prompt the user again and just assume that the answer to all further
questions is Yes.  Recompile causes all of \build\ to be recompiled even if
the compilation is not really necessary.
The user will then be asked if the \build\ environment definitions should
be loaded;  these (or equivalent) definitions must be loaded if \build\
is to be run.

\section{Running \build}

Once \build\ has been loaded into the environment, the following macros and
functions are available:
\begin{itemize}
\item {\tt defmodel}
\item {\tt build-request}
\item {\tt define-module-type}
\item {\tt define-process-type}
\item {\tt define-request-handler}
\item {\tt define-reference-handler}
\item {\tt define-reference}
\item {\tt access}
\item {\tt access*}
\item {\tt access+}
\item {\tt process-request}
\item {\tt touch}
\item {\tt forget}
\end{itemize}
Except for {\tt touch} and {\tt forget}, the functions and macros listed
above are documented in the \build\ technical report (forthcoming) and
the \build\ thesis paper.  In addition to the two \build\ papers, examples
of how several of the functions are to be used may be found in the files
{\tt test-models.lisp, standard.lisp,} and {\tt c-standard.lisp}.

The {\tt touch} function can be used in order to cause the creation
date attribute of a file to be updated to the present time without
actually manipulating the file.  This is useful for testing \build\ and
using the sample models.

While \build\ is running it maintains a table of information about the
modules that it caused to be created; the {\tt forget} function can
be used in order to cause \build\ to forget that some module was created.
Like {\tt touch}, this function is useful when testing \build\ and using the
sample models.

When \build\ is invoked it places a form on the current {\em logout list\/}
that causes the contents of the \build\ module table to be stored in a file
when the current user logs out.  \build\ will automatically load this
file the next time the user runs \build.  The data file allows \build\
to remember exactly which modules were used by the processes that it invoked.

Note: The \build\ papers refer to {\em g-nodes} while the source code refers to
{\em m-nodes}.  These two names are synonymous.  The term {\em m-node} has been
replaced with the term {\rm g-node,} but the prototype system has not been
updated to reflect the change.
\end{document}
